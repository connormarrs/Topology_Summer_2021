%%%%%%%%%%%%%%%%%%%%%%%%%%%%%%%%%%%%%%%%%%%%%%%%%%%%%%%%%%%%%%%%%%%%%%%%%%%%%%%%%%%%%%%%%%%%%%%%%%%%%%
%
% Problem Set 1: Take from Dennis Aroux's Math 131 Topology Course at Harvard University
% Written by Connor Marrs on June 11th, 2021
%
%%%%%%%%%%%%%%%%%%%%%%%%%%%%%%%%%%%%%%%%%%%%%%%%%%%%%%%%%%%%%%%%%%%%%%%%%%%%%%%%%%%%%%%%%%%%%%%%%%%%%%
\documentclass[./main.tex]{subfiles}
\graphicspath{{\subfix{../images/}}}

%%%%%%%%%%%%%%%%%%%%%%%%%%%%%%%%%%%%%%%%%%%%%%%%%%%%%%%%%%%%%%%%%%%%%%%%%%%%%%%%%%%%%%%%%%%%%%%%%%%%%%
%					Document Section Body
%%%%%%%%%%%%%%%%%%%%%%%%%%%%%%%%%%%%%%%%%%%%%%%%%%%%%%%%%%%%%%%%%%%%%%%%%%%%%%%%%%%%%%%%%%%%%%%%%%%%%%
%Adding your name here lets you make sure every page has your name, so that your psets don't get mixed up.
\begin{document}
%%%%%%%%%%%%%%%%%%%%%%%%% PROBLEM 1.1 %%%%%%%%%%%%%%%%%%%%%%%%
\begin{fprob}
    Recall that a subset U of a metric space is open if, for all $x\in U$, there exists some
    $\varepsilon >0$ such that $B_{\varepsilon}(x)$. Show that any open ball $B_r(p)$ is in fact an open set.
\end{fprob}
\begin{proof}
    Let $(X,d)$ be a metric space and let $p\in X$. Now, consider $B_\varepsilon(p)$ for some $\varepsilon>0$.
    Moreover, choose some $v\in B_\varepsilon(p)$, so by definition $d(p,v)<\varepsilon$. Now, let
    \[
        \gamma = \frac{1}{2}(\epsilon-d(p,v))
    \]
    and let $w\in B_\gamma(v)$; I claim that $B_\gamma(v)\subseteq B_\varepsilon(p)$. First, note that $\gamma>0$ since 
    $d(p,v)<\varepsilon$ by the inclusion $v\in B_\varepsilon(p)$. By the triangle inequality, we know that 
    $d(p,w)\leq d(p,v)+d(v,w)$, but $d(v,w)<\frac{1}{2}|\varepsilon-d(p,v)|$ by the inclusion $w\in B_\gamma(v)$. 
    We conclude that $d(p,w)\leq \frac{1}{2}\varepsilon+\frac{1}{2}d(p,v)$. Moreover, $d(p,v)<\varepsilon$, so 
    $d(p,w)<\frac{1}{2}\varepsilon+\frac{1}{2}\varepsilon=\varepsilon$. Now, it follows that $w\in B_\varepsilon(p)$, so the claim holds. 
\end{proof}

%%%%%%%%%%%%%%%%%%%%%%%%% PROBLEM 1.2 %%%%%%%%%%%%%%%%%%%%%%%%
\begin{fprob}
    Show that the collection of open sets in a metric space X satisfies the axioms of a topology, namely:
    \begin{enumerate}
        \item $\emptyset$ and $X$ are open;
        \item if $\{U_\alpha\}_{\alpha\in A}$ is an arbitrary collection of open sets, 
        then $\bigcup_{\alpha\in A}U_\alpha$ is open;
        \item if $U_1, \dots, U_n$ are open, then the intersection $\bigcap_{i=1}^nU_i$ is open.
    \end{enumerate}
    Finally, show with three examples that the arbitrary intersection of open sets is not necesarily open.
\end{fprob}

\begin{proof}
    First, we show that $\emptyset, X$ are open. Since the proposition $p\in X$ is always false by definition, the
    proposition "$x\in \emptyset$ $\Rightarrow$ there exists some $\varepsilon>0$ so that $B_\varepsilon(x)\subseteq\emptyset$" 
    must always be true. It follows that $\emptyset$ is open. Moreover, for any $p\in X$, we know that for any 
    $\varepsilon>0$, $B_\varepsilon(p)\subseteq X$ by definition.

    Now, we show that open sets are closed under arbitrary unions. Let $\{U_\alpha\}_{\alpha\in A}$ be a collection of
    open sets in $X$, and define $U = \bigcup_{\alpha\in A}U_\alpha$. Moreover, take $x\in U$, so for some 
    $\alpha\in A$, $x\in U_\alpha$. $U_\alpha$ is not just an ordinary set, though; it is open. That means there 
    is some $\varepsilon>0$ such that $B_\varepsilon(x)\subseteq U_\alpha$. We also know from set theory, though, 
    that this inclusion implies that $B_\varepsilon(x)\subseteq U$, so we conclude that $U$ is open. This shows that $(ii)$ holds.

    Finally, we show that open sets are closed under finite intersections. To whit, suppose that $S$ is a finite set, 
    and let $\{U_i\}_{i\in S}$ be a finite collection of open sets indexed by $S$. Now, define $U=\bigcap_{i\in S}U_i$ and
    let $x\in U$. Now, $x\in U_i$ for each $i\in S$, but each of these sets is open, so for every $i\in S$ there must be some 
    $\varepsilon_i>0$ so that $B_{\varepsilon_i}(x)\subseteq U_i$. Now, let $\varepsilon = \min_{i\in S}\{\varepsilon_i\}$.
    Now, take $v\in B_\varepsilon(x)$, so $d(x,v)<\varepsilon$. Moreover, $\varepsilon\leq \varepsilon_i$ for all $i\in S$,
    so $d(x,v)<\varepsilon_i$ for all $i\in S$. In other words, $v\in B_{\varepsilon_i}(x)$ for all $i\in S$, so $v\in U_i$ 
    for all $i\in S$. Now, $v\in \bigcap_{i\in S}=U$. Restating what we have shown, we see that $B_{\varepsilon}(x)\subseteq U$,
    so $U$ is open. This completes the proof of $(iii)$, and by extension, our solution.
\end{proof}

%%%%%%%%%%%%%%%%%%%%%%%%% PROBLEM 1.3 %%%%%%%%%%%%%%%%%%%%%%%%
\begin{fprob}
    \begin{enumerate}[(a)]
        \item Show that a sequence $\{p_i\}_{i\in\mn}$ in a metric space converges 
        to a limit $p$ if and only if every open set containing $p$ also contains $p_i$ 
        for all but finitely many $i$.
        \item Show that a sequence $\{p_i\}_{i\in\mn}$ in a metric space has at most one limit.
    \end{enumerate}
\end{fprob}

\begin{proof}
    %TODO:
\end{proof}

%%%%%%%%%%%%%%%%%%%%%%%%% PROBLEM 1.4 %%%%%%%%%%%%%%%%%%%%%%%%
\begin{fprob}
    Show that, if a sequence in a metric space converges to a limit, then it is a Cauchy
    sequence. Show (by giving an example) that the converse is not true.
\end{fprob}

\begin{proof}
    %TODO:
\end{proof}

%%%%%%%%%%%%%%%%%%%%%%%%% PROBLEM 1.5 %%%%%%%%%%%%%%%%%%%%%%%%
\begin{fprob}
    Given any two points $p = (p_1,\dots,p_n)$ and $q = (q_1,\dots, q_n)$ in $\mr^n$, define
    \[
        d_\infty(p,q) = \max\{|p_i-q_i|, i\in\mn_n\}  
    \]
    \begin{enumerate}[(a)]
        \item Show that $(\mr^n,d_\infty)$ is a metric space.
        \item Show that a subset of $\mr^n$ is open in $(\mr^n,d_\infty)$ if and only if it is open 
        for the Euclidean distance $(\mr^n,d)$. (In other terms: $d$ and $d_\infty$ induce the same 
        topology on $\mr^n$).
    \end{enumerate}
\end{fprob}

\begin{proof}
    %TODO:
\end{proof}


%%%%%%%%%%%%%%%%%%%%%%%%% PROBLEM 1.6 %%%%%%%%%%%%%%%%%%%%%%%%
\begin{fprob}
    Let $X$ be any set. For $p,q\in X$ and define $d(p,q)=\begin{cases} 1 & p\neq q \\ 0 & p=q\end{cases}$.
    Prove that this is a metric. Which subsets of the metric space $(X,d)$ are open, and which are closed?
\end{fprob}

\begin{proof}
    %TODO:
\end{proof}


\end{document}
