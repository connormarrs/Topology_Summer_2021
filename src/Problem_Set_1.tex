%%%%%%%%%%%%%%%%%%%%%%%%%%%%%%%%%%%%%%%%%%%%%%%%%%%%%%%%%%%%%%%%%%%%%%%%%%%%%%%%%%%%%%%%%%%%%%%%%%%%%%
%
% Problem Set 1: Take from Dennis Aroux's Math 131 Topology Course at Harvard University
% Written by Connor Marrs on June 11th, 2021
%
%%%%%%%%%%%%%%%%%%%%%%%%%%%%%%%%%%%%%%%%%%%%%%%%%%%%%%%%%%%%%%%%%%%%%%%%%%%%%%%%%%%%%%%%%%%%%%%%%%%%%%
\documentclass[./main.tex]{subfiles}
\graphicspath{{\subfix{../images/}}}

%%%%%%%%%%%%%%%%%%%%%%%%%%%%%%%%%%%%%%%%%%%%%%%%%%%%%%%%%%%%%%%%%%%%%%%%%%%%%%%%%%%%%%%%%%%%%%%%%%%%%%
%					Document Section Body
%%%%%%%%%%%%%%%%%%%%%%%%%%%%%%%%%%%%%%%%%%%%%%%%%%%%%%%%%%%%%%%%%%%%%%%%%%%%%%%%%%%%%%%%%%%%%%%%%%%%%%
%Adding your name here lets you make sure every page has your name, so that your psets don't get mixed up.
\begin{document}
%%%%%%%%%%%%%%%%%%%%%%%%% PROBLEM 1.1 %%%%%%%%%%%%%%%%%%%%%%%%
\begin{fprob}
    Recall that a subset U of a metric space is open if, for all $x\in U$, there exists some
    $\varepsilon >0$ such that $B_{\varepsilon}(x)$. Show that any open ball $B_r(p)$ is in fact an open set.
\end{fprob}
\begin{proof}
    Let $(X,d)$ be a metric space and let $p\in X$. Now, consider $B_\varepsilon(p)$ for some $\varepsilon>0$.
    Moreover, choose some $v\in B_\varepsilon(p)$, so by definition $d(p,v)<\varepsilon$. Now, let
    \[
        \gamma = \frac{1}{2}(\epsilon-d(p,v))
    \]
    and let $w\in B_\gamma(v)$; I claim that $B_\gamma(v)\subseteq B_\varepsilon(p)$. First, note that $\gamma>0$ since 
    $d(p,v)<\varepsilon$ by the inclusion $v\in B_\varepsilon(p)$. By the triangle inequality, we know that 
    $d(p,w)\leq d(p,v)+d(v,w)$, but $d(v,w)<\frac{1}{2}|\varepsilon-d(p,v)|$ by the inclusion $w\in B_\gamma(v)$. 
    We conclude that $d(p,w)\leq \frac{1}{2}\varepsilon+\frac{1}{2}d(p,v)$. Moreover, $d(p,v)<\varepsilon$, so 
    $d(p,w)<\frac{1}{2}\varepsilon+\frac{1}{2}\varepsilon=\varepsilon$. Now, it follows that $w\in B_\varepsilon(p)$, 
    so the claim holds. 
\end{proof}

%%%%%%%%%%%%%%%%%%%%%%%%% PROBLEM 1.2 %%%%%%%%%%%%%%%%%%%%%%%%
\begin{fprob}
    Show that the collection of open sets in a metric space X satisfies the axioms of a topology, namely:
    \begin{enumerate}
        \item $\emptyset$ and $X$ are open;
        \item if $\{U_\alpha\}_{\alpha\in A}$ is an arbitrary collection of open sets, 
        then $\bigcup_{\alpha\in A}U_\alpha$ is open;
        \item if $U_1, \dots, U_n$ are open, then the intersection $\bigcap_{i=1}^nU_i$ is open.
    \end{enumerate}
    Finally, show with three examples that the arbitrary intersection of open sets is not necesarily open.
\end{fprob}

\begin{proof}
    First, we show that $\emptyset, X$ are open. Since the proposition $p\in X$ is always false by definition, the
    proposition "$x\in \emptyset$ $\Rightarrow$ there exists some $\varepsilon>0$ so that $B_\varepsilon(x)\subseteq\emptyset$" 
    must always be true. It follows that $\emptyset$ is open. Moreover, for any $p\in X$, we know that for any 
    $\varepsilon>0$, $B_\varepsilon(p)\subseteq X$ by definition.

    Now, we show that open sets are closed under arbitrary unions. Let $\{U_\alpha\}_{\alpha\in A}$ be a collection of
    open sets in $X$, and define $U = \bigcup_{\alpha\in A}U_\alpha$. Moreover, take $x\in U$, so for some 
    $\alpha\in A$, $x\in U_\alpha$. $U_\alpha$ is not just an ordinary set, though; it is open. That means there 
    is some $\varepsilon>0$ such that $B_\varepsilon(x)\subseteq U_\alpha$. We also know from set theory, though, 
    that this inclusion implies that $B_\varepsilon(x)\subseteq U$, so we conclude that $U$ is open. This shows that $(ii)$ holds.

    Finally, we show that open sets are closed under finite intersections. To whit, suppose that $S$ is a finite set, 
    and let $\{U_i\}_{i\in S}$ be a finite collection of open sets indexed by $S$. Now, define $U=\bigcap_{i\in S}U_i$ and
    let $x\in U$. Now, $x\in U_i$ for each $i\in S$, but each of these sets is open, so for every $i\in S$ there must be some 
    $\varepsilon_i>0$ so that $B_{\varepsilon_i}(x)\subseteq U_i$. Now, let $\varepsilon = \min_{i\in S}\{\varepsilon_i\}$, and 
    take $v\in B_\varepsilon(x)$; then $d(x,v)<\varepsilon$. Moreover, $\varepsilon\leq \varepsilon_i$ for all $i\in S$,
    so $d(x,v)<\varepsilon_i$ for all $i\in S$. In other words, $v\in B_{\varepsilon_i}(x)$ for all $i\in S$, so $v\in U_i$ 
    for all $i\in S$. Now, $v\in \bigcap_{i\in S}=U$. Restating what we have shown, we see that $B_{\varepsilon}(x)\subseteq U$,
    so $U$ is open. This completes the proof of $(iii)$, and by extension, our solution.
\end{proof}

%%%%%%%%%%%%%%%%%%%%%%%%% PROBLEM 1.3 %%%%%%%%%%%%%%%%%%%%%%%%
\begin{fprob}
    \begin{enumerate}[(a)]
        \item Show that a sequence $\{p_i\}_{i\in\mn}$ in a metric space converges 
        to a limit $p$ if and only if every open set containing $p$ also contains $p_i$ 
        for all but finitely many $i$.
        \item Show that a sequence $\{p_i\}_{i\in\mn}$ in a metric space has at most one limit.
    \end{enumerate}
\end{fprob}

\begin{proof}
    (a, $\Rightarrow$) First, suppose that $\{p_i\}_{i\in\mn}$ is a sequence in a metric space $(X,d)$ where $p_i\rightarrow p$, 
    and let $U$ be an open set containing $p$. Since $U$ is open, there is $\varepsilon>0$ such that $B_\varepsilon(p)\subseteq U$, 
    and since $p_i\rightarrow p$, there must be an $N\in \mn$ such that if $n>N$, then $d(p_n,p)<\varepsilon$. In other words, if
    $n>N$, then $p_n\in B_\varepsilon(p)$, but this means that $p_n\in U$ as $B_\varepsilon(p)\subseteq U$. Taking the contrapositive 
    of this statement, we see that if $p_i\not\in U$, then $i<N$, and so only finitely many such $i$ can exist.

    ($\Leftarrow$) Now, let $(X,d)$ be a metric space with $p\in X$ and $\{p_i\}_{i\in\mn}$ a sequence therein. Moreover, suppose that
    for any open set $U$ containing $p$, there are only finitely many $i$ such that $p_i\not\in U$. Let $\varepsilon>0$, and consider 
    $B_\varepsilon(p)$, which by our reasoning above is open. Then, there are only finitely many $i\in \mn$ such that 
    $p_i\not\in B_\varepsilon(p)$, so let $S=\{i\in\mn | p_i\not\in B_\varepsilon(p)\}$. As $S$ is finite, it must have a maximum 
    element, say $N$. If $n>N$, then $n\not\in S$, since such an inclusion would contradict the assumption that $N$ was the maximum of 
    $S$. Note however, that this is the $N$ required by the definition of convergence, so $p_i\rightarrow p$.

    (b) Let that $(X,d)$ is a metric space and $\{p_i\}_{i\in\mn}$ a sequence therein. Suppose that for $p\neq q$, $p_i\rightarrow p$
    and $p_i\rightarrow q$. Since $p\neq q$, we know that $d(p,q)>0$, so if we take $r=\frac{d(p,q)}{3}$, then we can consider $U=B_d(p)$
    and $V=B_d(q)$. I claim that $U\cap V=\emptyset$. Note that
    \begin{align*}
	d(p,v)+d(v,q) &> d(p,q) \text{ by the triangle inequality}\\
	d(q,v) &> d(p,q) - d(p,v) \text{ rearranging terms}\\
	&> d(p,q) - d \text{ since $v\in B_d(p)$}\\
	&> 2d > d \text{ since $d=\frac{d(p,q)}{3}$}
    \end{align*}
    so we see that $d(q,v) > d$, so $q\not\in B_d(p)$, which proves our claim. By the result of (a) and the fact that $p_i\rightarrow p$,
    the set $S = \{i\in \mn | p_i\not\in B_d(p)\}$ is finite. Similarly, we can reason that the set $R =\{i\in\mn| p_i\in B_d(q)\}$ is 
    infinite. Moreover, $B_d(q)\subseteq X\setminus B_d(p)$ by our reasoning abover, so $R\subseteq S$. Since any subset of a finite set
    must be finite, we have a contradiction. Thus, the assumption that distinct $p$ and $q$ can exist must be false, so we conclude
    that limits are unique in metric spaces.
\end{proof}

%%%%%%%%%%%%%%%%%%%%%%%%% PROBLEM 1.4 %%%%%%%%%%%%%%%%%%%%%%%%
\begin{fprob}
    Show that, if a sequence in a metric space converges to a limit, then it is a Cauchy
    sequence. Show (by giving an example) that the converse is not true.
\end{fprob}

\begin{proof}
	Let $(X,d)$ be a metric space and $\{p_i\}_{i\in\mn}$ be a sequence therein such that $p_i\rightarrow p$. Fix some $\varepsilon>0$,
    	and since $p_i\rightarrow p$, we know that there is some $N\in\mn$ such that if $n>N$, then $p_i\in B_{\frac{\varepsilon}{2}}(p)$.
	Moreover, for any $x,y\in B_{\frac{\varepsilon}{2}}(p)$,
	\begin{align*}
		d(x,y) &< d(x,p)+d(y,p) \text{ by the triangle inequality}\\
		&< \frac{\varepsilon}{2} + \frac{\varepsilon}{2} = \varepsilon
	\end{align*}
	Now, we see that if $m,n>N$, $d(p_n,p_m)<\varepsilon$, so $\{p_i\}_{i\in\mn}$ is indeed Cauchy.

	To see that the converse is true, consider $\{\sum_{i=1}^n \frac{4(-1)^i}{2i+1}\}_{n\in\mn}$. We can see that this is a sequence in 
	$(\mq,d)$, and it is a result from analysis that this sequence converges to $\pi$. Thus, in $(\mr, d)$, it converges, so it is 
	cauchy by the above reasoning. By keeping the same metric, we see that this sequence must also be cauchy in $(\mq, d)$. Since limits
    are unique in metric spaces, this sequence cannot converge to any other value in $\mr$, meaning it cannot converge to a value in $\mq$.
\end{proof}

%%%%%%%%%%%%%%%%%%%%%%%%% PROBLEM 1.5 %%%%%%%%%%%%%%%%%%%%%%%%
\begin{fprob}
    Given any two points $p = (p_1,\dots,p_n)$ and $q = (q_1,\dots, q_n)$ in $\mr^n$, define
    \[
        d_\infty(p,q) = \max\{|p_i-q_i|, i\in\mn_n\}  
    \]
    \begin{enumerate}[(a)]
        \item Show that $(\mr^n,d_\infty)$ is a metric space.
        \item Show that a subset of $\mr^n$ is open in $(\mr^n,d_\infty)$ if and only if it is open 
        for the Euclidean distance $(\mr^n,d)$. (In other terms: $d$ and $d_\infty$ induce the same 
        topology on $\mr^n$).
    \end{enumerate}
\end{fprob}

\begin{proof}
    (a) First, we show that $d_\infty$ is symmetric. Let $x,y\in \mr^n$, and let $x_i$ or $y_i$ denote the ith coordinate 
    of $x$ or $y$ respectively. Since $|x_i-y_i|=|y_i-x_i|$, we can see that 
    \begin{align*}
    	d_\infty(x,y) &= \max_{i\in\mn_n}|x_i-y_i| \\
	    &= \max_{i\in\mn_n}|y_i-x_i| = d_\infty(y,x)
	\end{align*}
	Now, we show that $d_\infty$ obeys the triangle inequality. Let $x,y,z\in \mr^n$, and let $i=\arg\max_{i\in\mn_n}\{|x_i-y_i\}$.
    It follows that
    \begin{align*}
        d_\infty(x,z) + d_\infty(y,z) &= \max_{j\in\mn_n}\{|x_j-z_j\} + \max_{j\in\mn\{|y_j-z_j\}}\\
        &\geq |x_i-z_i| + |y_i-z_i| \text{ by the definition of a maximum}\\
        &\geq |x_i-y_i| \text{ by the triangle inequality for absolute value}\\
        &= d_\infty(x,y)
    \end{align*}
    so the triangle inequality does hold. Finally, we will show that $d_\infty(x,y)=0$ if and only if $x=y$. First, suppose that
    $d_\infty(x,y)=0$ for $x,y\in \mr^n$. Note that each $|x_i-y_i|\geq 0$ by definition, but 
    \[
        0 = d_\infty(x,y)=\max_{j\in\mn_n}\{|x_j-y_j|\} \geq |x_i-y_i|
    \]
    for any $i\in\mn_n$, so $|x_i-y_i|=0$ for any $i$. This implies, though, that $x=y$. Moreover, if $x=y$, then $x_i=y_i$ for
    any $i\in\mn_n$, so $|x_i-y_i| =0$, meaning $d_\infty(x,y)=0$. We conclude that $d_\infty$ is a metric on $\mr^n$.

    (b) First, I claim that for all $x,y\in \mr^n$, 
    \[
        d_\infty(x,y) \leq d(x,y) \leq \sqrt{n} d_\infty(x,y)
    \]
    Provided that the above claim holds, consider some open subset $U\subset\mr^n$ that is open in the Euclidean topology. 
    Let $B'$ denotes an open ball in the $d_\infty$ topology and $B_\varepsilon(x)$ denote an open Euclidean ball. Now,
    for any $p\in B'_{\frac{\varepsilon}{\sqrt{n}}}(x)$, we see that
    \begin{align*}
        d(x,p) \leq \sqrt{n}d(x,y) < \sqrt{n}\frac{\varepsilon}{\sqrt{n}} = \varepsilon
    \end{align*}
    so $B'_{\frac{\varepsilon}{\sqrt{n}}}(x)\subseteq B_\varepsilon(x)\subseteq U$, so $U$ is also open in the $d_\infty$.
    Conversely, consider some $U\subseteq\mr^n$ that is open in the $d_\infty$ topology. Now, for any $x\in U$, there is 
    $\varepsilon>0$ such that $B'_{\varepsilon}(x)\subseteq U$. Moreover, for any $p\in B'_{\varepsilon}(x)$, we have that
    \begin{align*}
        d(x,p) &\leq d_\infty(x,p) <\varepsilon
    \end{align*}
    so $p\in B_{\varepsilon}(x)$. Now, we have that $B_\varepsilon(x)\subseteq B'_{\varepsilon}(x)\subseteq U$, so $U$ is open
    in the Euclidean topology as well. To finish the proof, we now justify our claim.
    \medskip

    We start with the first equality; let $x,y\in\mr^n$, and let $j=\arg\max_{i\in\mn_n}\{|x_i-y_i\}$. We know that
    \begin{align*}
        d_\infty(x,y) = |x_j-y_j| &= \sqrt{(x_j-y_j)^2}\\
        &\leq \sqrt{\sum_{i=1}^n(x_i-y_i)^2} = d(x,y)
    \end{align*}
    where the last inequality holds since the sum is clearly larger than one of its terms, and square roots preserve order (this
    can be seen by considering the contrapositive of "$x>y$ $\Rightarrow$ $x^2>y^2$"). Now, we consider the second inequality. By the
    definition of $j$, we know that for any $i\in\mn$, $|x_i-y_i|\leq |x_j-y_j|$. By induction, we can see that 
    \[
        \sum_{i=1}^n(x_i-y_i)^2 \leq \sum_{i=1}^n(x_j-y_j)^2
    \]
    Combining this fact with the result that square roots preserve relative order, we conclude that
    \begin{align*}
        d(x,y)=\sqrt{\sum_{i=1}^n(x_i-y_i)^2} &\leq \sqrt{n(x_j-y_j)^2}\\
        & = \sqrt{n}|x_j-y_j| = \sqrt{n}d_\infty(x,y)
    \end{align*}
    Thus, our claim holds, and we see that indeed a subset of $\mr^n$ is open in the Euclidean topology if and only if it is open
    in the infinity metric.
\end{proof}


%%%%%%%%%%%%%%%%%%%%%%%%% PROBLEM 1.6 %%%%%%%%%%%%%%%%%%%%%%%%
\begin{fprob}
    Let $X$ be any set. For $p,q\in X$ and define $d(p,q)=\begin{cases} 1 & p\neq q \\ 0 & p=q\end{cases}$.
    Prove that this is a metric. Which subsets of the metric space $(X,d)$ are open, and which are closed?
\end{fprob}

\begin{proof}
    First, we show that the discrete metric is symmetric. Suppose that $x,y\in\mr^n$, and note that either $x=y$ or $x\neq y$.
    If, on the one hand $x=y$, then $d(x,y)=0$, but we also know that $y=x$, so $d(y,x)=0$. Thus, $d(x,y)=d(y,x)$. If on the
    other hand, $x\neq y$, then $d(x,y)=1$, and as before $y\neq x$, so $d(y,x)=1$, meaning $d(x,y)=d(y,x)$. In either case,
    however, the discrete metric is symmetric.
    \medskip

    Now, we show that the metric is definite. If $x,y\in\mr^n$ where $d(x,y)=0$, we can see that $x=y$ by considering the
    contrapositive of the statement $x\neq y$ $\Rightarrow$ $d(x,y)=1$.
    \medskip

    Finally, we show that the triangle inequality holds. Again, take $x,y,z\in\mr^n$. First, suppose that $x=y=z$, in which
    case $d(x,y)=d(y,z)=d(x,z)=0$, so 
    \begin{align*}
        d(x,y) + d(y,z) = 0+0 = 0 &= d(x,z)\\ 
        d(y,x) + d(x,z) = 0+0 = 0 &= d(y,z)\\
        d(x,z) + d(z,y) = 0+0 = 0 &= d(x,y)\\
    \end{align*}
    meaning that the triangle inequality holds. On the other
    hand, suppose that precisely two of the points coincide, and suppose without loss of generality that $x=y\neq z$. Then,
    \begin{align*}
        d(x,y) + d(y,z) = 0+1 \geq 0 &= d(x,z)\\ 
        d(y,x) + d(x,z) = 0+1 \geq 0 &= d(y,z)\\
        d(x,z) + d(z,y) = 1+1 \geq 0 &= d(x,y)\\
    \end{align*}
    so the triangle inequality holds. Finally, suppose that all three points are distinct. In this case, for any relabeling of $x,y,z$,
    we can see that
    \[
        d(x,y)+d(y,z)= 1+1\geq 1 = d(x,z)
    \]
    so again the triangle inequality holds. We conclude that the discrete metric is indeed worthy of its name. Moreover, we can easily
    see that for any $x\in X$, $\{x\}$ is open by considering $B_{\frac{1}{2}}(x)=\{x\}$. Now, any singleton is open, but arbitrary unions
    of open sets are open (by the result of the first problem), and since any set is the union of its singletons, any subset of $X$ is open.
    Moreover, the complement of any open set in $X$ is also a set in $X$, so it must be open. This tells us that any subset $U\subseteq X$ is
    both open and closed in the discrete metric. 
\end{proof}


\end{document}
